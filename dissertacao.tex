\documentclass[12pt,a4paper]{ufpr2} %ufpr2.cls cont�m o co-orientador, j� o ufpr.cls cont�m apenas o orientador

% \usepackage[portuges,brazil]{babel}
% \usepackage[portuguese,brazil]{babel}
\usepackage[brazil]{babel}
\usepackage[latin1]{inputenc}
\usepackage{amssymb,amsmath}
\usepackage{epsfig}
\usepackage{multirow}



%\usepackage{isolatin1}

\usepackage{ifthen,graphicx,color}
\graphicspath{{figuras/}}
\usepackage[latin1]{inputenc}
%\usepackage{auto-pst-pdf}
%\usepackage{auto-pst-pdf}

\usepackage{amssymb}
\usepackage{subfigure}
\usepackage{graphicx}
%\usepackage{caption2}
\usepackage{setspace}
\usepackage{ps-macros}

\usepackage[brazil]{babel}
\usepackage[latin1]{inputenc}
\usepackage[T1]{fontenc}

\usepackage{adjustbox}
\usepackage{blindtext}
\usepackage{amssymb,amsmath}
\usepackage{epsfig}
\usepackage{multirow}
\usepackage{ifthen,graphicx,color}
\usepackage{amssymb}
\usepackage{subfigure}
\usepackage{graphicx}
\usepackage{setspace}
\usepackage{ps-macros}

\usepackage[onelanguage,portuguese,algochapter,linesnumbered,ruled,inoutnumbered]{algorithm2e} % adicionado onelanguage e portuguese
\graphicspath{{./Imagens/}}

%=========== Extras - n�o existentes no modelo original ===========
\usepackage[shortcuts]{extdash}
\usepackage[bottom]{footmisc}
\usepackage{isolatin1}
\usepackage[hypcap]{caption}
\usepackage{tabularx}
\usepackage{tabulary}
\usepackage[table]{xcolor}
\usepackage{float}
\usepackage{booktabs}
\usepackage{multirow}
\usepackage{rotating}
\usepackage{pdflscape}
\usepackage{algorithm2eportuguese}
\DeclareGraphicsExtensions{.png,.jpg}
\linespread{1.3}               % espa�amento entre linhas
    
 % penalidades de quebra de p�gina
\widowpenalty=10000           
\clubpenalty=10000
%=========== Extras - n�o existentes no modelo original ===========

\setcounter{secnumdepth}{3}    % n - numero de niveis de subsubsection numeradas
\setcounter{tocdepth}{3}       % coloca ate o nivel n no sumario

\title{T�tulo do Trabalho}
\author{Jackson Antonio do Prado Lima}
\advisortitle{Orientadora} % ou Orientador
\advisorname{Prof.$^a$ Dr.$^a$ Silvia Regina Vergilio}
\advisorplace{Departamento de Inform�tica, UFPR}  % departamento, instituicao

%=========== Co-Orientador ===========
% se usar o ufpr2.cls, presente no in�cio do documento, utilizar as defini��es abaixo, caso contr�rio deve coment�-las
\coadvisortitle{Co-orientadora} % ou Orientador
\coadvisorname{Me. Giovani Guizzo}
\coadvisorplace{Departamento de Inform�tica, UFPR}  % departamento, instituicao
%=========== Co-Orientador ===========

\city{Curitiba}
\year{2015}

\banca  % nao insira o nome do orientador, ja eh feito automaticamente
{}{}    % se nao houver deixe em branco {}{}
{}{}    % se nao houver deixe em branco {}{}
{}{}    % se nao houver deixe em branco {}{}
{}{}    % se houver um quarto membro na banca, inserir nome e instituicao

\defesa{25 de dezembro de 2015} % dia em que foi realizada a defesa da dissertacao

\begin{document}

%\makecapaproposta             % cria capa para proposta%
\makecapadissertacao           % cria capa para dissertacao de mestrado %
\makerosto                     % cria folha de rosto para versao final da UFPR %
%\maketermo                    % cria folha com o termo de aprovacao da dissertacao%

%\singlespacing           % espacamento 1 - capa UFPR%
%\onehalfspacing          % espacamento 1/2 %
\doublespacing            % espacamento 2 - UFPR %

\pagestyle{headings}
\pagenumbering{roman}

%\chapter*{Agradecimentos}
%\input{agradecimentos.tex}          % possiu somente o texto

\tableofcontents

%\listoffigures         % se houver mais do que 3 figuras
%\addcontentsline{toc}{chapter}{\MakeUppercase{Lista de Figuras}}
%\newpage

%\listoftables        % se houver mais do que 3 tabelas
%\addcontentsline{toc}{chapter}{\MakeUppercase{Lista de Tabelas}}
%\newpage

\chapter*{Resumo}
\addcontentsline{toc}{chapter}{\MakeUppercase{Resumo}}
Texto do resumo....
           % somente o texto
\newpage

\chapter*{Abstract}
\addcontentsline{toc}{chapter}{\MakeUppercase{Abstract}}
Texto do abstract....
        % somente o texto
\newpage


\pagenumbering{arabic}

\chapter{Introdu\c{c}\~ao}
\label{Introducao}

Teste para a introdu��o da disserta��o, refer�ncia\cite{tese1,artigo1}

\section{Novo}

Teste para a introdu��o da disserta��o, refer�ncia\cite{tese1,artigo1}



% *****************
% O [41] determina o percentual de reducao em relacao ao tamanho original.
% *****************
%\begin{figure}
%\centerfig{feature_space_2D.ps}[41]
%\caption{Legenda geral da figura.}
%\label{figura_xpto}
%\end{figure}
% *****************



Teste para a introdu��o da disserta��o, refer�ncia\cite{tese1,artigo1}
Teste para a introdu��o da disserta��o, refer�ncia\cite{tese1,artigo1}
Teste para a introdu��o da disserta��o, refer�ncia\cite{tese1,artigo1}
Teste para a introdu��o da disserta��o, refer�ncia\cite{tese1,artigo1}
Teste para a introdu��o da disserta��o, refer�ncia\cite{tese1,artigo1}
Teste para a introdu��o da disserta��o, refer�ncia\cite{tese1,artigo1}




% *****************
% O .50 da minipage e' para dividir a largura da pagina em 2 figuras por
% linhas, se for colocar 3 subfiguras por linha .33 e assi vai...
% a largura da imagem e' 3cm.(width=3cm).
%
% O (!ht) e' para forcar o latex a colocar a figura na posicao, ou
% paragrafo onde foi inserido no texto o \begin{figure}, "se possivel"
% e' claro...
% *****************
%\begin{figure}[!ht]
%\renewcommand{\captionfont}{\it}
%\renewcommand{\captionlabelfont}{\bf}
%\begin{minipage}[b]{.50\textwidth}
%\centering
%  \subfigure[Imagem 1]{
%    \label{fig:teste:a}
%    \includegraphics[width=3cm]{feature_space_2D.ps}}
%\end{minipage}%
%\begin{minipage}[b]{.50\textwidth}
%\centering
%  \subfigure[Imagem 2]{
%    \label{fig:teste:b}
%    \includegraphics[width=3cm]{feature_space_2D.ps}}
%\end{minipage}
%\caption{
%  Legenda geral da figura contendo 2 sub-figuras colocadas lado a lado.}
%\label{fig:teste}
%\end{figure}
% *****************


Teste para a introdu��o da disserta��o, refer�ncia\cite{tese1,artigo1}
Teste para a introdu��o da disserta��o, refer�ncia\cite{tese1,artigo1}
Teste para a introdu��o da disserta��o, refer�ncia\cite{tese1,artigo1}
Teste para a introdu��o da disserta��o, refer�ncia\cite{tese1,artigo1}
Teste para a introdu��o da disserta��o, refer�ncia\cite{tese1,artigo1}
Teste para a introdu��o da disserta��o, refer�ncia\cite{tese1,artigo1}


% *****************
%\begin{figure}
%\psfig{file=feature_space_2D.ps,height=2in,width=3.5in}
%\caption{Legenda geral da figura usando psfig COM reducao.}
%\end{figure}
% *****************

Teste para a introdu��o da disserta��o, refer�ncia\cite{tese1,artigo1}
Teste para a introdu��o da disserta��o, refer�ncia\cite{tese1,artigo1}
Teste para a introdu��o da disserta��o, refer�ncia\cite{tese1,artigo1}
Teste para a introdu��o da disserta��o, refer�ncia\cite{tese1,artigo1}
Teste para a introdu��o da disserta��o, refer�ncia\cite{tese1,artigo1}
Teste para a introdu��o da disserta��o, refer�ncia\cite{tese1,artigo1}


% *****************
%\begin{figure}
%\psfig{file=feature_space_2D.ps}
%\caption{Legenda geral da figura usando psfig SEM reducao.}
%\end{figure}
% *****************

%\input{capitulo2.tex}
%\input{capitulo3.tex}
%\input{capitulo4.tex}
%\input{capitulo5.tex}
%\input{capitulo6.tex}
%\input{anexo1.tex}     % se houver anexo

\bibliographystyle{brazil}
\bibliography{bibliografia}
% utilize macros (3 primeiras letras do mes em ingles, minusculas) no seu
% .bib para atribuir o nome do mes em portugues nas referencia,
% se o style for brazil, outros estilos tambem aceitam estas macros
% Ex:
%
% @InProceedings{teste,
%   author =       {Luciano}
%   year =         {2000}
%   month =        {}#sep;
% }
%
\addcontentsline{toc}{chapter}{\MakeUppercase{Bibliografia}}

\singlespacing
%\makecapadissertacao

\end{document}
